\documentclass[a4paper,10pt]{article}

\usepackage[utf8]{inputenc}  
\usepackage[T1]{fontenc}
\usepackage{graphicx}
\usepackage{tikz}
\usepackage[top=3cm, bottom=3cm, left=3cm, right=3cm]{geometry}
\usepackage{hyperref}

\usepackage{float}
\usepackage[francais]{babel}
\hypersetup{                    % parametrage des hyperliens
    colorlinks=true,                % colorise les liens
    breaklinks=true,                % permet les retours à la ligne pour les liens trop longs
    urlcolor= blue,                 % couleur des hyperliens
    linkcolor= black,                % couleur des liens internes aux documents (index, figures, tableaux, equations,...)
    citecolor= green                % couleur des liens vers les references bibliographiques
    }

\graphicspath{{pictures/}}
\DeclareUnicodeCharacter{00A0}{ }
\widowpenalty=10000
\clubpenalty=10000
\input{environnements.tex}
\setlength{\parindent}{0.5cm}
\begin{document}
 

\title{Projet Imagerie Multispectrale}
\author{Aurélien Barbotin, Pierre David, Benjamin Michelland, Youna Le Vaou}

%P'tit bout de code pour centrer le titre
\null  % Empty line
\nointerlineskip  % No skip for prev line
\vfill
\let\snewpage \newpage
\let\newpage \relax
\maketitle
\let \newpage \snewpage
\vfill 
\break % page break

\tableofcontents
\newpage

%\chapter{Présentation du projet}
\documentclass[a4paper,10pt]{article}
\usepackage[utf8x]{inputenc}
\usepackage{url}
\usepackage{graphicx}
\usepackage{array}

%opening
\title{Etat de lard}
\author{Aurélien Barbotin Pierre David Benjamin Michelland Youna le Vaou}

\begin{document}

\maketitle

\section{État de l'art}
\subsection{Imagerie satellite hyperspectrale}

Jusqu'à récemment, la principale source d'images satellite hyperspectrales exploitées par les scientifiques étaient les satellites MODIS (Moderate-Resolution Imaging SpectroRadiometer). Ces deux satellites, lancés par la NASA en 1999 et 2002, imagent l'intégralité de la surface de la terre tous les deux jours. Ces satellites sont capables de mesurer 36 bandes fréquentielles avec trois résolutions différentes : 250m, 500m et 1000m\cite{nasa}. L'objectif de cette mission est de ``jouer un rôle vital dans le développement de modèles globaux, validés et interactifs de systèmes terrestres capables de prévoir des changements globaux avec suffisamment de précision pour aider les décideurs politiques à prendre des décisions avisées concernant la protection de notre environnement.''

Ces données ont ainsi pu servir à analyser la qualité de l'air au niveau du sol et, entre autres, d'en déduire la présence d'évènements exceptionnels comme des feux de forêt ou du brouillard\cite{airSurv}. A l'aide d'indices comme le NDVI (normalized difference vegetation index) calculés à partir de données hyperspectrales, il est possible d'établir la présence ou l'absence de végétation dans une zone. En effet, les plantes chlorophyliennes absorbent fortement la lumière rouge et réfléchissent la lumière proche infrarouge : la différence normalisée d'intensité entre ces deux bandes correspond à l'indice NDVI dont une valeur proche de 1 indique la présence de végétation, et une valeur proche de 0 son absence. De multiples applications, comme l'étude de la désertification de zones dues à l'activité humaine\cite{desert} ont dores et déjà été mises en évidence.

La faible résolution des données MODIS ne permet en revanche pas de discerner des détails comme des cours d'eau ou des habitations, et des données plus précises sont donc nécessaires pour obtenir de meilleures classifications. 

Dans le cadre du projet \textit{Copernicus} d'observation de la Terre, l'agence spatiale européenne (ESA) développe en ce moment les missions Sentinel. Chacune de ces missions consiste en une paire de satellite en orbite autour de notre planète, récoltant des données sur la surface et l'atmosphère. Ainsi, les satellites Sentinel 2 (Sentinel 2-A lancé le 23 juin 2015 et 2-B dont le lancement est prévu pour la seconde moitié de 2016)\cite{sent2} récoltent des données hyperspectrales sur 13 bandes de fréquence : 4 bandes avec une résolution de 10m et 9 avec une résolution de 60m.

Les données Sentinel 2 étant très récentes, nous faisons partie des premiers à les analyser et aucun article à leur sujet n'a encore été publié à notre connaissance. Leur analyse et leur classification relève donc de la recherche.

\section{Résultats}
Dans un but de répétabilité, nous avons testé et optimisé toutes nos méthodes de machine learning sur une unique image hyperspectrale (dont une vue en RGB est donnée en figure \ref{fig:veniseRGB}).
\begin{figure}
  \centering
    \includegraphics[width=0.5\textwidth]{venise}
  \caption{Image hyperspectralde de Venise qui nous a servi de référence (vue en RGB)}
  \label{fig:veniseRGB}
\end{figure}
Le résultat d'une classification est une image RGB dont la couleur de chaque pixel représente la classe à laquelle il appartient selon l'algorithme. Par exemple, dans une image résultat, un pixel bleu correspond à un point indentifé par l'algorithme comme étant de l'eau. Notre code couleur est résumé dans le tableau \ref{table:codeCouleur}.
\begin{figure}
 \begin{center}
  \begin{tabular}{|c|c|}
    \hline
    Nature du terrain & couleur \\
    \hline
  Champ & Vert \\
  Ville &  Rouge \\
  Eau &  Bleu \\
  Boue & Marron \\
    \hline
  \label{table:codeCouleur}
  \end{tabular}
\caption{Code couleur des images produites par machine learning.} 
\end{center}
\end{figure}
Estimer la qualité d'un classificateur revient à estimer la qualité de l'image résultante. Pour cela, deux méthodes s'offrent à nous: la première numérique, consiste à calculer la matrice de confusion de chaque classificateur comme expliqué INSERER ICI OU CEST EXPLIQUE. L'autre méthode consiste à estimer à l'oeil la correspondance entre les prédictions de nos algorithmes et la réalité (les zones de référence étant elles-mêmes choisies à l'oeil, ce critère n'est pas plus mauvais que l'autre). Pour immédiatement visualiser cette correspondance, nous superposons l'image de base avec l'image résultante en transparence, comme par exemple sur la figure \ref{fig:veniseLSE}.

\subsection{classificateur linéaire}
\label{lineaire}
La première classification que nous testée est aussi la seule que nous avons entièrement implémentée nous mêmes. Il s'agit de classification à l'aide d'un classificateur linéaire. Les résultats obtenus sont encourageants, avec une première classification qui à première vue correspond à la répartition réelle des trois classes étudiées (figure \ref{fig:veniseLSE} avec la matrice de confusion correspondante \ref{table:confknn}).

\begin{figure}
  \centering
    \includegraphics[width=0.5\textwidth]{veniseLSE}
  \caption{Résutat de classification avec un classificateur linéaire}
  \label{fig:veniseLSE}
\end{figure}

\begin{figure}
\begin{center}
 \begin{tabular}{|c|c|c|c|c|}
  \hline
  Nature du terrain & Ville & Champ & Eau & Rappel \\
  \hline
Ville & 5309   &   878    &   15 & 0.856014 \\
Champ & 1296   &  9832     &   0  & 0.883537 \\
Eau &  2   &     0  &   6350 & 0.999685 \\
Précision & 0.803542  & 0.918021 & 0.997643 & 0.907483 \\
  \hline
\end{tabular}
\end{center}
\caption{Matrice de confusion de la classification linéaire}
\label{table:confknn}
\end{figure}

La précision totale de cette méthode est de 90.7\%. On peut constater que l'eau est très bien classifiée, et que les erreurs proviennent majoritairement de confusions entre les champs et la ville.

\subsection{LDA: Analyse linéaire de discriminant}
La classification LDA, pour Linear Discriminant Analysis (Analyse linéaire de discriminant) suppose une distribution gaussienne des points au sein de chacune des classes. Or, nous avons choisi des polygones de manière à ce qu'ils contiennent le moins de points possibles, le plus représentatifs possibles, afin de diminuer les temps de calcul. Ce faisant, on n'a pas pris un échantillon continu de points et l'approximation gaussienne est difficilement valable, ce qui explique les mauvais résultats obtenus, en particulier sur les zones de ville.
\begin{figure}
  \centering
    \includegraphics[width=0.5\textwidth]{venise+LDA}
  \caption{Résutat de classification avec une analyse linéaire de discriminant}
  \label{fig:veniseLDA}
\end{figure}

\begin{table}
\begin{center}
 \begin{tabular}{|c|c|c|c|c|}
  \hline
  Nature du terrain & Ville & Champ & Eau & Rappel \\
  \hline
Ville & 2026  &   4176  &      0 & 0.326669 \\
Champ & 675  &  10453   &     0 & 0.9393427 \\
Eau &  0    &    0  &   6352   &     1 \\
Précision & 0.750093 & 0.71454  &      1 & 0.795161\\
  \hline
\end{tabular}
\end{center}
\caption{Matrice de confusion de l'analyse avec discriminant linéaire }
\label{table:veniseLDA}
\end{table}
    
     

\subsection{méthode knn:k-nearest-neighbours ou k plus proches voisins}
La méthode des k-nearest neighbors présente des résultats intéressants, et nous avons obtenu une classification très précise avec cette méthode. Nous avons testé cette méthode avec différentes valeurs de k (1, 3, 20, 200) et il s'est avéré que cette valeur n'influait pas sur la classification, ce qui nous laisse penser que les points sont naturellement bien séparés. Nous avons donc choisi de faire notre classification avec la valeur du paramètre k=1. Les résultats obtenus sont montrés sur la figure \ref{fig:1NN} et le tableau \ref{table:1NN}.
\begin{table}
\begin{center}
 \begin{tabular}{|c|c|c|c|c|}
  \hline
  Nature du terrain & champ & ville & eau & Rappel \\
  \hline
Champ & 9929 & 71 & 	0 &	99.29 \\
Ville & 540 &	9460 &	0 &	94.6 \\
Eau &  0 &	0 &	10000 &	100 \\
Précision & 99.29 & 94.6 & 100 & 97.96 \\
  \hline
\end{tabular}
\end{center}
\label{table:1NN}
\caption{Matrice de confusion de l'algorithme de 1-plus proche voisin.}
\end{table}

\begin{figure}
  \centering
    \includegraphics[width=0.5\textwidth]{resultat1NN}
  \caption{Résutat de classification avec la méthode du 1-plus proche voisin}
  \label{fig:1NN}
\end{figure}

\subsection{Support Vecteur Machine}
Cette méthode utilise deux métaparamètres : C et $\gamma$. Afin d'optimiser l'apprentissage et la classification de notre image, il faut donc trouver les métaparamètres idéaux. Pour cela, on fait un ensemble de validations croisées dans lequel on sépare notre jeu de données en 5, 4 parts servant à l'apprentissage et la dernière part au test. On fait ce test 100 fois pour 100 couples de valeurs (C,$\gamma$), et le jeu de paramètres obtenant le meilleur score de précision lors de la validation croisée correspondaux paramètres que l'on utilisera. Afin de déterminer d'un seul coup d'oeil si l'on a trouvé un set de paramètres optimal, on représente les résultats des différentes validations croisées sur une courbe en deux dimensions (figure \ref{fig:crossMap}).

\begin{figure}
  \centering
    \includegraphics[width=0.5\textwidth]{crossValLog}
  \caption{Scores obtenus pouf 100 validations croisées avec différents couples de paramètres (C,$\gamma$). La couleur d'un point correspond à la précision moyenne de l'algorithme pour le couple (C,$\gamma$) qui lui sert de coordonnées. Les axes sont en échelle logarithmique.}
  \label{fig:crossMap}
\end{figure}

Cette figure qu'en effet, la précision de l'algorithme dépend du choix judicieux des métaparamètres. Le meilleur choix de métaparamètres dans notre cas est le couple (C,$\gamma$)=(1,$ 10^{-3}$). Nous obtenons alors l'image \ref{fig:veniseSVM}et la matrice de confusion \ref{table:SVC}.

\begin{table}
\begin{center}
 \begin{tabular}{|c|c|c|c|c|}
  \hline
  Nature du terrain & Ville & Champ & eau & Rappel \\
  \hline
Ville & 9867 & 133 & 	0 &	98.67 \\
Champ & 159 &	9841 &	0 &	98.41 \\
Eau &  38 &	0 &	9962 &	99,62 \\
Précision & 98.04 & 98.6 & 100 & 98.9 \\
  \hline
  \end{tabular}
\end{center}
\label{table:SVC}
\caption{Matrice de confusion de l'algorithme de Support Vecteur Machine.}
\end{table}

\begin{figure}
  \centering
    \includegraphics[width=0.5\textwidth]{veniseSVM}
  \caption{Résutat de classification avec le support vecteur machine.}
  \label{fig:veniseSVM}
\end{figure}

\subsection{Conclusion}
La première conclusion que nous pouvons tirer de notre travail est que les méthodes de machine learning que nous avons testées donnent pour la plupart des résultats très satisfaisants. Le support vecteur machine en particulier nous fournit une accuracy de 98.9\% avec trois classes. Il est d'usage de prétraiter les données pour calculer des indices comme l'index de végétation différentiel normalisé (NDVI)\cite{NDVI}:
\begin{equation}
NDVI=\frac{NIR-VIS}{NIR+VIS}
\end{equation}
Où NIR correspond à l'intensité lumineuse dans le proche infrarouge et VIS dans le rouge. Une valeur de cet indice proche de 1 indique la présence de végétation, une valeur négative indique des nages ou l'absence de végétation. Dans notre cas, les images multispectrales fournissent suffisamment d'informations par elles-mêmes pour classifier les sols avec précision sans passer par cet indice.

Nous avons également démontré que chacune des 12 bandes est nécessaire dans la classification, puisque la perte de l'une d'entre elles entraîne une baisse significative de l'accuracy comme on l'a vu dans le paragraphe \ref{lineaire}.

\subsection{Limites et améliorations}
Nous avons constaté que la boue le long des côtes n'est pas toujours reconnue de la même manière par nos algorithmes: parfois champs, parfois ville, ils ne correspondent à aucune de nos trois classes. Il est en effet rare en apprentissage automatique d'utiliser seulement trois classes. Nous avons donc rajouté une classe ``boue'' et testé certaines de nos classifications dessus, en particulier le Support Vecteur Machine. Le résultat est donné en figure \ref{fig:SVM4Cl}. La boue est bien classifiée mais l'introduction de cette 4e classe entraîne l'apparition de faux positifs, en particulier dans les champs dont la réponse spectrale est proche.
\begin{figure}
  \centering
    \includegraphics[width=0.5\textwidth]{SVM4Classes}
  \caption{Résutat de classification de 4 classes avec le support vecteur machine.}
  \label{fig:SVM4Cl}
\end{figure}

Plusieurs méthodes s'offrent à nous pour améliorer ce résultat de classification: la plus simple, appelée \textit{whitening} (blanchissement, en français) consiste à centrer chaque bande spectrale sur sa moyenne et à la dilater proportionnellement à son écart-type. Cette méthode renormalise les distributions et permet une meilleure classification.

Il est également possible d'ajouter encore plus d'informations, en utilisant les données de satellites Sentinel-1 qui sont des images radar. Une dernière méthode tout juste mise au point par notre tuteur N. Brodu consiste à améliorer la résolution des bandes spectrales de plus faible résolution en 

Enfin, les données sentinel-2 étant toutes nouvelles, peu de scènes sont à notre disposition. Nous n'avons donc que très peu testé nos algorithmes de classification sur des zones différentes de notre zone test. Il serait intéressant de vérifier que la classification fonctionne sur d'autres zones sans avoir besoin de redéfinir des polygones d'apprentissage.

\bibliography{Biblio}
\bibliographystyle{plain}
\end{document}


\section{Etat de l'Art}
\subsection{Travaux scientifiques}
%parler des articles qu'Aurélien a trouvé pour la soutenance

Les données multispectrales des satellites MODIS sont largement utilisées par la communauté scientifique. Ici on présentera trois publications qui utilisent différentes approches de la surveillance satellite.

\paragraph{}
L'analyse des données multispectrales peut se faire en calculant des indices particuliers, comme le NDVI (normalized difference vegetation index). Le NDVI permet d'établir la présence ou l'absence de végétation dans une zone. En effet, les plantes chlorophylliennes absorbent fortement la lumière rouge et réfléchissent la lumière proche infrarouge : la différence normalisée d'intensité entre ces deux bandes correspond à l'indice NDVI. 
\begin{equation}
NDVI=\frac{NIR-VIS}{NIR+VIS}
\end{equation}
Où NIR correspond à l'intensité lumineuse dans le proche infrarouge et VIS dans le rouge.\newline
Une valeur proche de 1 indique la présence de végétation, et une valeur proche de 0 son absence. L'utilisation de cet indice est très courante. Il a ainsi été utilisé pour l'étude de la désertification de zones dues à l'activité humaine\cite{desert} par Meng-Lung et al.

\begin{figure}[H]
  \centering
    \includegraphics[width=0.5\textwidth]{etudeDesertification.png}
  \caption{Étude de la désertification, image tirée de l'article \textit{Fuzzy model based assessment and monitoring of desertification using MODIS satellite imagery}, Meng-Lung Lin et al. (2009)}
  \label{fig:etudeDesert}
\end{figure}

\paragraph{}
En outre, les données satellites peuvent servir à analyser la qualité de l'air au niveau du sol et, entre autres, en déduire la présence d'évènements exceptionnels comme des feux de forêt ou du brouillard\cite{airSurv}. 

\begin{figure}[H]
  \centering
    \includegraphics[width=0.5\textwidth]{surveillanceAtmo.png}
  \caption{Suivi atmosphérique, image tirée de l'article \textit{Qualitative and quantitative evaluation of MODIS satellite sensor data for regional and urban scale air quality}, Engel-Cox et al. (2004)}
  \label{fig:survAtmo}
\end{figure}

\paragraph{}
Plus proche de notre projet, la classification peut se faire par des algorithmes de machine learning. Cette approche a été suivie dans l'étude menée par M.A. Friedl et al\cite{mapping}. 

\begin{figure}[H]
  \centering
    \includegraphics[width=0.75\textwidth]{classifSols.png}
  \caption{Classification des sols, image tirée de l'article \textit{Global land cover mapping from MODIS: algorithms and early results}, M.A. Friedl et al. (2002)}
  \label{fig:clSols}
\end{figure}

\subsection{Logiciels de classification}
Nous allons ici présenter quelques logiciels qui permettent de faire de la surveillance d'occupation des sols. Il en existe un très grand nombre, c'est pourquoi nous nous sommes concentrés sur les principaux logiciels utilisés dans le domaine.
\paragraph{QGIS} 
\paragraph{}
QGIS \footnote{\href{http://www.qgis.org}{Site Officiel de QGIS }}est un logiciel de cartographie libre publié sous licence GPL. Multiplateforme, il permet de traiter les formats usuels d'image satellite, mais aussi d'y ajouter des couches vectorielles pour délimiter des polygones ou classifier des zones géographiques.
De plus, la communauté de QGIS a développé de nombreux "plugins" (modules d'extension) permettant d'appliquer différents algorithmes de machine learning.\newline
Le logiciel SAGA, intégré à QGIS, utilise l'algorithme de ressemblance maximale, qui permet de faire une classification statistique des pixels. Ayant choisi des polygones d'entraînement, ils vont être assimilés à des lois normales et à partir de leurs moyennes et de leurs variances, les pixel inconnus vont appartenir à la classe à laquelle ils ont le plus de chance d'appartenir.
Semi-Automatic Classification Plugin, permet également d'obtenir des classifications à partir d'image à l'aide de différents algorithmes.
OrfeoToolBox est un autre logiciel qui a la possibilité d'être utilisé via QGIS et qui peut réaliser une classification d'images satellite.

\paragraph{ENVI}
\paragraph{}
ENVI (ENvironment for Visualizing Images) \footnote{\href{http://www.exelisvis.fr/ProduitsetServices/LesproduitsENVI/ENVI.aspx}{Site officiel d'ENVI}} est un logiciel propriétaire, payant, sous licence commerciale. Il permet de traiter efficacement les données satellites à l'aide de plusieurs algorithmes dont celui de la ressemblance maximum. C'est un logiciel très utilisé dans l'industrie et qui est relativement facile d'utilisation.
\paragraph{ArcGIS}
\paragraph{}
Parmi les logiciels payant sous licence propriétaire, on peut aussi noter ArcGIS \footnote{\href{http://www.esrifrance.fr/arcgis.aspx}{Site officiel d'ArcGIS}}, développé par la société Esri (Environmental Systems Research Institute, Inc.). Il contient également une boîte à outil de traitement des images géographiques relativement complète.

\section{Méthode et organisation}
\subsection{Processus}
\paragraph{}
L'objectif de ce projet était de classifier une zone géographique à partir des images satellites fournies par Sentinel-2. Par souci de lisibilité, nous avons choisi de présenter les résultats sous la forme d'images RGB composées d'autant de teintes qu'on a choisi de classes (par exemple pour une classification tri-classes, l'image finale a trois teintes : rouge, verte et bleue).
\paragraph{}
Afin d'arriver à ce résultat, il a fallu traiter les données fournies par l'ESA. Celles-ci ont donc subi un processus résumé ci-dessous : \\\\
\includegraphics[scale=0.3]{process_Multispec.pdf}
\paragraph{SNAP}
\paragraph{}
SNAP\footnote{\href{http://step.esa.int/main/toolboxes/snap/}{SNAP | STEP}} (ou Sentinel Application Platform) est un logiciel open-source et gratuit développé par l'Agence Spatiale Européenne (ESA). Il est le seul capable d'ouvrir les données de Sentinel-2 et d'afficher les différentes bandes de l'image multispectrale. Pour pouvoir manipuler simplement l'image (l'originale étant très volumineuse et dans un format complexe et inadapté), nous avons donc sélectionné une sous-image que nous avons exportée au format GéoTIFF.
\paragraph{QGIS}
\paragraph{}
Afin de pouvoir utiliser les algorithmes d'apprentissage automatique, nous avons besoin de sélectionner des pixels dont on connaît par avance la classe (urbain, champ ou eau). Sous QGIS, nous avons donc tracé des polygones autour de zones ne contenant qu'un seul type de classe : des polygones autour d'une région ne contenant que de l'eau, puis que des champs, puis que de la ville. Nous avons ensuite exporté les coordonnées de ces polygones.
\paragraph{Script Python}
\paragraph{}
Comme il a été précédemment écrit, pour pouvoir utiliser les algorithmes d'apprentissage automatique, il faut choisir sur l'image des pixels dont on connaît par avance la classe. Nous avons donc écrit un script sous Python en utilisant l'image sélectionnée et les coordonnées des polygones sélectionnés auparavant. Nous avons extrait les valeurs des 13 bandes spectrales de tous les pixels contenus à l'intérieur des polygones. Ces "pixels" de 13 bandes formeront les vecteurs d'entraînement de nos algorithmes d'apprentissage automatique. On enregistre alors trois listes de pixels sous format texte : une liste des pixels "urbain", une autre contenant les pixels "champs" et enfin une dernière liste avec les pixels "eau".
\paragraph{Sk-learn, dlib et PRT}
\paragraph{}
Sk-learn\footnote{\href{http://scikit-learn.org}{scikit-learn.org}}, dlib\footnote{\href{http://dlib.net}{dlib.net}} et PRT\footnote{\href{https://github.com/covartech/PRT}{github.com/covartech/PRT}} sont trois bibliothèques permettant d'utiliser des algorithmes d'apprentissage automatique. Elles sont respectivement publiées sous Licence BSD, Licence Logicielle Boost et Licence MIT, toutes trois des licences de logiciel libre. Toutes les trois sont écrites dans un langage différent, respectivement Python, C++ et MatLab. L'idée était de tester les différentes implémentations, de comparer leur rapidité et leur facilité d'utilisation. Nous avons utilisé quatre algorithmes classiques pour la classification : la méthode aux moindres carrés, celle des K-Nearest Neighbors, l'Analyse Discriminante Linéaire et des Machines à Vecteur de Support. Les résultats seront exposés et comparés plus loin dans le rapport. Nous avons donc utilisé différents algorithmes sous différentes implémentations sur notre image. Finalement, nous avons créé une image RGB avec en rouge les zones urbaines, en vert les zones agricoles et en bleu l'eau. Associé à cette image, nous avons aussi calculé une matrice de confusion, outil mathématique permettant d'évaluer rapidement et finement la qualité de la classification opérée par un des algorithmes d'apprentissage automatique.
\subsection{Organisation du projet}
\paragraph{Répartition du travail}
\paragraph{}
Dans un premier temps, il a fallu tous nous familiariser avec les notions d'apprentissage automatique et avec les différents logiciels à utiliser. Nous avons donc tous les quatre travaillé en commun. Chacun a donc appris à utiliser SNAP, puis QGIS. Quand nous avons commencé à mieux comprendre le projet et à être plus à l'aise, nous nous sommes partagés les tâches.
\paragraph{}
Pendant que Youna, Aurélien et Pierre continuaient de se documenter sur l'apprentissage automatique, Benjamin travaillait sur le script Python permettant l'extraction des pixels "intéressants". C'est donc Benjamin qui a codé ce bloc. Une fois cette partie terminée, à l'aide des pixels extraits, nous avons tous pu commencer à implémenter les algorithmes d'apprentissage automatique. Chacun a choisi un langage différent : Benjamin a choisi le C++, Aurélien le Python, Youna et Pierre le langage MatLab. Nous avons tous codé entièrement la méthode des moindres carrés car celle-ci est la méthode la plus simple et qu'elle permet de comprendre tout le principe de la classification. Cette première méthode a donc eu un intérêt pédagogique.
\paragraph{}
C'est à partir de ce moment que nous avons commencé à paralléliser les tâches : chacun a implémenté les algorithmes d'apprentissage automatique avec différents langages et différentes bibliothèques : Benjamin a utilisé dlib sous C++, Aurélien sk-learn sous Python, Youna ANN (Approximate Nearest Neighbors, publiée sous Licence LGPL) sous C++ pour coder la méthode des k plus proches voisins et Pierre a utilisé PRT sous MatLab.
\paragraph{Mise en commun du travail}
\paragraph{}
Afin d'organiser au mieux le code que nous avons écrit, nous avons utilisé Git. Cela nous a permis de partager nos différents codes ainsi que nos différents résultats. Par ailleurs, nous avons choisi d'écrire notre rapport sous LaTeX. Git s'est donc aussi révélé très utile quand il s'est agi de mettre en commun les contributions de chacun dans la rédaction. À ce propos, nous avons décidé de nommer un coordinateur pour la rédaction du rapport : Pierre et un autre coordinateur pour la soutenance : Youna.
\paragraph{Rencontres avec le tuteur}
\paragraph{}
Le projet s'est intégralement structuré autour de réunions hebdomadaires avec notre tuteur Nicolas Brodu. Ces réunions avaient à la fois pour but de rapporter les progrès que nous avions faits mais aussi de nous apprendre les bases de Machine Learning. Ces réunions ont donc été capitales dont la compréhension du sujet et donc dans l'avancement du projet. Le projet ayant duré 6 semaines, nous avons fait 6 réunions. Voici donc l'historique des réunions et les points abordés.
\paragraph{04/11 :} Dans un premier temps, Nicolas Brodu nous a exposé ce sur quoi il travaillait, il a aussi dressé un court état de l'art de la classification des sols. Il nous a ensuite proposé plusieurs sujets portant sur ce domaine et nous avons alors choisi de développer l'aspect Apprentissage Automatique avec pour objet les images multispectrales de Sentinel-2. Une fois ce choix fait, nous avons établi avec lui un cahier des charges.
\paragraph{10/11 :} L'enjeu de la première semaine était de se familiariser avec les images multispectrales, d'apprendre à se servir de SNAP (ouvrir une image, extraire une sous-image et l'exporter au bon format) de QGIS (ouvrir la sous image, tracer et extraire des polygones autour de zones d'intérêt). C'est aussi au cours de cette semaine que nous avons commencé à coder le script Python pour extraire les valeurs des pixels contenus dans les zones d'intérêt. Nous avons donc rapporté à notre tuteur notre avancement et nos difficultés quant à l'écriture du script. M. Brodu a également profité de cette réunion pour nous apprendre les principes de l'apprentissage automatique et plus particulièrement du  Machine Learning  linéaire (notamment la classification aux moindres carrés).
\paragraph{18/11 :} C'est à partir de cette réunion que nous avons pu montrer les premiers résultats de classification à notre tuteur. Nous avons également pu montrer que la bande 10 était inutilisable sur l'image que nous avions prise pour travailler et que les 12 autres bandes étaient nécessaires (en enlevant une ou plusieurs bandes la précision se dégradait rapidement à l'œil nu). M. Brodu nous alors introduit la matrice de confusion, outil permettant d'évaluer quantitativement la justesse de classification. Il a également exposé d'autres algorithmes d'apprentissage automatique : la LDA\footnote{\href{https://en.wikipedia.org/wiki/Linear_discriminant_analysis}{Linear Discriminant Analysis}} (Analyse discriminante linéaire) et la PCA\footnote{\href{https://en.wikipedia.org/wiki/Principal_component_analysis}{Principal Component Analysis}} (Analyse en Composantes Principales). Face au temps qu'il nous restait, il était exclu d'implémenter nous même chaque algorithme d'apprentissage automatique. C'est pourquoi nous avons utilisé des bibliothèques contenant déjà les algorithmes. Notre tuteur nous a suggéré plusieurs bibliothèques, toutes sous licences libres. À propos de MatLab, c'est effectivement un logiciel payant et non libre. Nous avons malgré tout continué à l'utiliser en sachant que s'il fallait passer à une alternative libre, nous pourrions toujours utiliser Octave\footnote{\href{https://www.gnu.org/software/octave/}{GNU Octave}}, qui a l'avantage d'avoir presque la même syntaxe que MatLab.
\paragraph{25/11 :} Nous avons eu peu de nouveaux résultats à rapporter cette semaine : nous nous sommes concentrés sur l'apprentissage des méthodes des différentes bibliothèques. Nous avons quand même pu confirmer grâce à l'utilisation de la matrice de confusion que la méthode des moindres carrés fonctionnait surprenamment bien pour une méthode aussi simpliste. Si la méthode la plus basique de classification fonctionne aussi bien, quelle précision pouvons nous espérer avec des méthodes plus complexes ? Nous avons voulu apporter une réponse à cette question et c'est pourquoi M. Brodu nous a expliqué les principe du Machine Learning non linéaire en commençant par développer deux méthodes : les SVM\footnote{\href{https://en.wikipedia.org/wiki/Support_vector_machine}{Support Vector Machine}} (Machines à Vecteurs de Support) et la méthode des KNN\footnote{\href{https://en.wikipedia.org/wiki/K-nearest_neighbors_algorithm}{K-Nearest Neighbors}} (K plus Proches Voisins). Ces algorithmes ont la particularité de contenir des méta-paramètres (par exemple K pour les KNN), qu'il faut bien choisir. Notre tuteur nous a donc donné la méthode classique pour les choisir : faire de la validation croisée.
\paragraph{09/12 :} Cette réunion a été la dernière où nous avons pu exposer nos résultats. Nous avons exposé tout ce que nous avions fait et le reste de la réunion a porté sur les possibles ouvertures du projet : Faire du blanchiment (pour attribuer le même poids aux différentes bandes dans la classification), utiliser des algorithmes de super-résolution, utiliser les bandes radar de Sentinel-1, utiliser d'autres indicateurs. M. Brodu a ainsi évoqué la Malédiction de la dimensionalité (utiliser toujours plus d'indicateurs peut dégrader des résultats si l'on ne fait pas attention).
\paragraph{16/12 :}Cette dernière réunion a constitué en une soutenance blanche, nous avons fait une première présentation à notre tuteur et celui-ci nous a aidé à l'améliorer et à mieux expliquer l'apprentissage automatique à des personnes ne connaissant pas ce domaine.

\subsection{Innovation et valorisation possible du projet}
\paragraph{}
Si nous avions pu continuer notre projet et le valoriser en un projet d'entreprise, nous aurions cherché à rentabiliser notre expertise plutôt que de protéger notre travail par des mesures directes de propriété intellectuelles. En effet, les images utilisées sont libres et accessibles par tous et, bien que les bibliothèques utilisées soient sous licences libres, nous n'avons fait qu'utiliser les fonctions principales de machine learning : il n'y a donc pas beaucoup de code à valoriser. En revanche, nous avons acquis une expérience et une certaine expertise de l'utilisation des méthodes de machine learning pour la classification des sols. Nous sommes parmi les premiers à avoir tester ces algorithmes sur les données Sentinel-2 et cette longueur d'avance, si elle est maintenue, peut nous donner un avantage concurrentiel important. Ainsi, une façon de valoriser notre projet serait de vendre notre expertise à des acteurs de l'industrie agricole, minière ou pétrolière, de manière plus générale à toutes les entreprises utilisant la surveillance des sols dans leurs activités. Nous vendrions donc nos services aux entreprises souhaitant sous-traiter cette étape de leur activité. Un des avantages de n'avoir utilisé que des bibliothèques libres et gratuites serait alors de ne pas avoir de coût d'exploitation sur celles-ci et de pouvoir compter sur une communauté libriste dynamique améliorant fréquemment les différentes bibliothèques. Ainsi, la valorisation de notre projet par la propriété intellectuelle serait indirecte.

\section{Résultats}
Le résultat d'une classification est une image RGB dont la couleur de chaque pixel représente la classe à laquelle il appartient selon l'algorithme. Par exemple, dans une image résultat, un pixel bleu correspond à un point identifié par l'algorithme comme étant de l'eau. Notre code couleur est résumé dans le tableau \ref{table:codeCouleur}.
\begin{figure}[H]
 \begin{center}
  \begin{tabular}{|c|c|}
    \hline
    Nature du terrain & couleur \\
    \hline
  Champ & Vert \\
  Ville &  Rouge \\
  Eau &  Bleu \\
  Boue & Marron \\
    \hline
  \end{tabular}
\caption{Code couleur des images produites par machine learning.} 
\label{table:codeCouleur}
\end{center}
\end{figure}
Estimer la qualité d'un classificateur revient à estimer la qualité de l'image résultante. Pour cela, deux méthodes s'offrent à nous: la première numérique, consiste à calculer la matrice de confusion de chaque classificateur comme expliqué en 1.1.3. Pour plus de clarté, l'exactitude sera en rouge. L'autre méthode consiste à estimer à l'œil la correspondance entre les prédictions de nos algorithmes et la réalité (les zones de référence étant elles-mêmes choisies à l'œil, ce critère n'est pas plus mauvais que l'autre). Pour immédiatement visualiser cette correspondance, nous superposons l'image de base avec l'image résultante en transparence, comme par exemple sur la figure \ref{fig:veniseLSE}.

 \subsection{Les classifications linéaires}
 
 Réaliser une classification linéaire d'un ensemble de données en différentes classes revient, en deux dimensions, à trouver la droite qui sépare au mieux deux ensembles de vecteurs. Dans la figure \ref{fig:ml_lin}, on prend l'exemple d'un ensemble de points représentant de l'eau et de l'urbain. Sur l'image, les points représentant l'eau auront une couleur plutôt bleue alors que les points représentant des zones urbaines seront plutôt rouge, voire gris. Ils sont assez bien distingués pour trouver une séparation linéaire.

\begin{figure}[H]
  \centering
    \includegraphics[width=0.5\textwidth]{ml_lin.png}
  \caption{Classification linéaire}
  \label{fig:ml_lin}
\end{figure}

Il existe plusieurs méthodes pour trouver une droite qui sépare correctement les deux ensembles de points.

\paragraph{La méthode des moindres carrés}
\paragraph{}
  La première méthode consiste à faire l'hypothèse que tous les points des deux ensembles sont alignés, et qu'on peut alors trouver une droite telle que tous les points soient à une distance de 1, pour l'une des deux classes, et de -1 pour l'autre. On se ramène alors à une optimisation linéaire qui peut être résolue par la méthode des moindres carrés.
  C'est la première classification que nous avons testée et aussi la seule que nous avons entièrement implémentée nous-mêmes. 
  
  \begin{figure}[H]
  \centering
    \includegraphics[width=0.75\textwidth]{ml_lse}
  \caption{Classification linéaire}
  \label{fig:ml_lse}
\end{figure}
  
\label{lineaire}
 Les résultats obtenus sont encourageants, avec une première classification qui à première vue correspond à la répartition réelle des trois classes étudiées (figure \ref{fig:veniseLSE} avec la matrice de confusion correspondante \ref{table:confLSE}).

\begin{figure}[H]
  \centering
    \includegraphics[width=0.75\textwidth]{veniseLSE}
  \caption{Résultat de classification avec un classificateur linéaire}
  \label{fig:veniseLSE}
\end{figure}

\begin{figure}[H]
\begin{center}
 \begin{tabular}{|c|c|c|c|c|}
  \hline
  Nature du terrain & Eau & Champ & Urbain & Rappel \\
  \hline
Eau & 10000   &   0    &   0 & 100\% \\
Champ & 0   &  9991     &   9  & 99.9\% \\
Urbain &  11   &     2712  &   7277 & 72.8\% \\
Précision & 99.9\%  & 78.9\% & 99.9\% & {\color{red}90.9\%} \\
  \hline
\end{tabular}
\end{center}
\caption{Matrice de confusion de la classification linéaire}
\label{table:confLSE}
\end{figure}

L'exactitude de cette méthode est de 90.9\%. On peut constater que l'eau est très bien classifiée, et que les erreurs proviennent majoritairement de confusions entre les champs et la ville. Il est également intéressant de voir que cette confusion n'est pas symétrique : si lez zones urbaines sont classifiées comme des champs par l'algorithme, l'inverse est beaucoup moins vrai.

\paragraph{Analyse discriminante linéaire ou de Fisher}
\paragraph{}
  Une seconde méthode consiste à utiliser l'analyse discriminante linéaire\footnote{ou analyse discriminante de Fisher}(LDA), c'est-à-dire poser l'hypothèse que chaque ensemble de points a une distribution gaussienne, et à trouver, à partir de la variance et de la moyenne de ces distributions la meilleure séparation entre les ensembles. On peut voir cette méthode comme une recherche de droite sur laquelle la projection des gaussiennes est la mieux séparé, la meilleure séparation est alors l'hyperplan perpendiculaire à cette droite. Cependant, dans la pratique, l'hypothèse de distribution gaussienne est très forte et rarement vérifiée.

\begin{figure}[H]
  \centering
    \includegraphics[width=0.4\textwidth]{ml_lda}\hfill
    \includegraphics[width=0.5\textwidth]{ml_ldaReel}
  \caption{Analyse discriminante linéaire : hypothèse (à g.) vs repartition réelle (à d.)}
  \label{fig:ml_lda}
\end{figure}

  En effet, nous avons choisi des polygones de manière à ce qu'ils contiennent le moins de points possibles, les plus représentatifs possibles, afin de diminuer les temps de calcul. Ce faisant, nous n'avons pas pris un échantillon continu de points et l'approximation gaussienne est difficilement valable, ce qui explique les mauvais résultats obtenus, en particulier sur les zones de ville.
\begin{figure}[H]
  \centering
    \includegraphics[width=0.75\textwidth]{venise+LDA}
  \caption{Résultat de classification avec une analyse linéaire de discriminant}
  \label{fig:veniseLDA}
\end{figure}

\begin{table}[H]
\begin{center}
 \begin{tabular}{|c|c|c|c|c|}
  \hline
  Nature du terrain & Eau & Champ & Urbain & Rappel \\
  \hline
Eau & 9966  &   0  &      34 & 99.7\% \\
Champ & 0  &  10000   &     0 & 100\% \\
Urbain &  0    &    6461  &   3539   &     35.4\% \\
Précision & 100\% & 60.7\%  &      99.0\% & {\color{red}78.4\%}\\
  \hline
\end{tabular}
\end{center}
\caption{Matrice de confusion de l'analyse avec discriminant linéaire }
\label{table:veniseLDA}
\end{table}
    
     

\subsection{Les classifications non-linéaires}

Dans certains cas, les ensembles sont impossibles à classifier de manière linéaire. Il faut alors utiliser des algorithmes non linéaires. Dans la figure \ref{fig:ml_nlin}, on souhaite classifier un ensemble de points représentant l'eau et les champs. Les points représentant l'eau sont toujours plutôt bleus, mais les points représentant les champs peuvent varier du vert au jaune, voire des teintes rouges. Les données sont alors beaucoup plus compliquées à classifier efficacement de manière linéaire. 

\begin{figure}[H]
  \centering
    \includegraphics[width=0.5\textwidth]{ml_nlin}
  \caption{Classification non linéaire}
  \label{fig:ml_nlin}
\end{figure}

\paragraph{Machine à Vecteurs de Support (SVM) :}
\paragraph{}
Les SVM ont été développées dans les années 1990 d'après les travaux de Vladimir Vapnik. Cette approche généralise les classificateurs linéaires en cherchant une séparation linéaire de marge maximum. Pour cela, l'idée est dans un premier temps de chercher les vecteurs des différentes classes qui définissent le mieux l'enveloppe de chaque classe, et de trouver l'hyperplan qui maximise l'écart à ces vecteurs de support.

\begin{figure}[H]
  \centering
    \includegraphics[width=0.5\textwidth]{ml_svm}
  \caption{Classification par SVM}
  \label{fig:ml_svm}
\end{figure}

Pour augmenter la précision de la classification, il est intéressant de trouver une frontière non-linéaire. Pour ce faire, on utilise l'astuce du noyau (ou kernel trick). Le \textit{kernel trick}\cite{aizermanSVM} consiste à remplacer le produit scalaire de l'espace considéré par l'évaluation d'une fonction (appelée noyau), on peut ramener l'étude d'un problème non séparable linéairement, à un problème linéaire. La classification admet alors des méta-paramètres, qui devront être fixés par l'utilisateur avant de réaliser l'entraînement. Afin d'optimiser l'apprentissage et la classification de notre image, il faut trouver les méta-paramètres idéaux. Pour cela, on utilise une méthode d'évaluation dite validation croisée.

\subparagraph{La validation croisée\newline}
Il s'agit d'une technique d'évaluation de la classification qui va nous permettre de maximiser la taille des polygones d'entraînement et de test. En effet, cette technique consiste à n'avoir qu'un seul jeu de polygones qui seront utilisés intégralement pour le test et pour l'entraînement. L'astuce réside ici à utiliser une partie des éléments du polygone pour l'entraînement et le reste pour le test, puis de réitérer en changeant les éléments utilisés pour le test et ainsi de suite jusqu'à ce que tous les éléments aient été utilisé pour le test.
\begin{center}
\renewcommand{\arraystretch}{4}
\begin{tabular}{c | c || c || c}
  Etape 1& Etape 2 & Etape 3\\
  \cellcolor{Periwinkle}test & \cellcolor{NavyBlue}entrainement & \cellcolor{NavyBlue}entrainement\\
  \cellcolor{NavyBlue}entrainement & \cellcolor{Periwinkle}test & \cellcolor{NavyBlue}entrainement\\
  \cellcolor{NavyBlue}entrainement & \cellcolor{NavyBlue}entrainement & \cellcolor{Periwinkle}test\\
\end{tabular}
  \captionof{figure}{Illustration de la validation croisée.}
\end{center}


Dans notre cas, on utilise deux méta-paramètres : C et $\gamma$. on sépare notre jeu de données en 5, 4 parts servant à l'apprentissage et la dernière part au test. On fait ce test 100 fois pour 100 couples de valeurs (C,$\gamma$), et le jeu de paramètres obtenant le meilleur score de précision lors de la validation croisée correspond aux paramètres que l'on utilisera. Afin de déterminer visuellement si l'on a trouvé un set de paramètres optimal, on représente les résultats des différentes validations croisées sur une courbe en deux dimensions (figure \ref{fig:crossMap}).

\begin{figure}[H]
  \centering
    \includegraphics[width=0.5\textwidth]{crossValLog}
  \caption{Scores obtenus pouf 100 validations croisées avec différents couples de paramètres (C,$\gamma$). La couleur d'un point correspond à la précision moyenne de l'algorithme pour le couple (C,$\gamma$) qui lui sert de coordonnées. Les axes sont en échelle logarithmique.}
  \label{fig:crossMap}
\end{figure}

Cette figure montre qu'en effet, la précision de l'algorithme dépend du choix judicieux des méta-paramètres. Le meilleur choix de méta-paramètres dans notre cas est le couple (C,$\gamma$)=(1,$ 10^{-3}$). Nous obtenons alors l'image \ref{fig:veniseSVM} et la matrice de confusion \ref{table:SVC}.

\begin{table}[H]
\begin{center}
 \begin{tabular}{|c|c|c|c|c|}
  \hline
  Nature du terrain & Eau & Champ & Urbain & Rappel \\
  \hline
Eau & 9999 & 0 & 	1 &	100\% \\
Champ & 0 &	9991 &	9 &	99.9\% \\
Urbain &  0 &	244 &	9756 &	97.6\% \\
Précision & 100\% & 97.6\% & 99.9\% & {\color{red}99.1\%} \\
  \hline
  \end{tabular}
\end{center}
\caption{Matrice de confusion de l'algorithme de Support Vecteur Machine.}
\label{table:SVC}
\end{table}

\begin{figure}[H]
  \centering
    \includegraphics[width=0.75\textwidth]{veniseSVM}
  \caption{Résultat de classification avec le support vecteur machine.}
  \label{fig:veniseSVM}
\end{figure}


\paragraph{Les K plus proches voisins :}
\paragraph{}
La méthode des k plus proches voisins part d'une idée assez simple. Considérons un ensemble de points, dont les caractéristiques et les classes sont connues. Considérons un nouvel élément à classer. L'idée est qu'il sera de la même classe que le ou les points de caractéristiques les plus proches. Cette méthode présente donc aussi un méta-paramètre : le nombre de voisins k à prendre en compte. 

\begin{figure}[H]
  \centering
    \includegraphics[width=1.1\textwidth]{ml_knn}
  \caption{Classification k plus proches voisins}
  \label{fig:ml_knn}
\end{figure}

Nous avons testé cette méthode avec différentes valeurs de k (1 à 10, 20, 200) et il s'est avéré que les plus faibles valeurs de k sont aussi celles qui fournissent la meilleure exactitude (voir figure \ref{fig:kNN}), ce qui nous laisse penser que les points sont naturellement bien séparés. Nous avons donc choisi de faire notre classification avec la valeur du paramètre k=1. Les résultats obtenus sont montrés sur la figure \ref{fig:1NN} et le tableau \ref{table:1NN}.

La méthode des k-nearest neighbors présente des résultats intéressants, et nous avons obtenu une classification très précise avec cette méthode. 

\begin{table}[H]
\begin{center}
 \begin{tabular}{|c|c|c|c|c|}
  \hline
  Nature du terrain & Eau & Champ  & Urbain & Rappel \\
  \hline
Eau & 9999 & 0 & 	1 &	100\% \\
Champ & 0 &	9929 &	71 &	99.3\% \\
Urbain &  0 &	540 &	9460 &	94.6\% \\
Précision & 100\% & 94.8\% & 99.2\% & {\color{red}98.0\%} \\
  \hline
\end{tabular}
\end{center}
\caption{Matrice de confusion de l'algorithme de 1-plus proche voisin.}
\label{table:1NN}
\end{table}

\begin{figure}[H]
  \centering
    \includegraphics[width=0.75\textwidth]{resultat1NN}
  \caption{Résultat de classification avec la méthode du 1-plus proche voisin}
  \label{fig:1NN}
\end{figure}

On observe également une exactitude en moyenne moindre pour les valeurs paires de k, par rapport aux valeurs impaires : cela vient du fait que nous n'avons pas mis en place de système de vote en cas de conflit. En effet, si la moitié des voisins appartiennent à une classe, et l'autre moitié appartient à une autre classe, le système choisit aléatoirement dans quelle classe placer le pixel. Il est possible de remédier à cela en choisissant un système de vote adapté (par exemple, en cas d'égalité, calculer la distance à chaque voisin et choisir la classe pour laquelle la distance totale est la plus faible).

\begin{figure}[H]
  \centering
    \includegraphics[width=0.5\textwidth]{influencek}
  \caption{Influence du paramètre k sur l'exactitude de l'algorithme.}
  \label{fig:kNN}
\end{figure}

\subsection{Conclusion}

\paragraph{Des performances intéressantes}
\paragraph{}
La première conclusion que nous pouvons tirer de notre travail est que \textbf{les méthodes de machine learning que nous avons testées donnent pour la plupart des résultats très satisfaisants}. Le support vecteur machine en particulier nous fournit une exactitude de \textbf{99.1\%} avec trois classes. A l'état de l'art, et pour notre problématique, une classification est satisfaisante pour une exactitude supérieure à 95\%.

\paragraph{Pas de calcul d'indice nécessaire}
\paragraph{}
Il est d'usage de prétraiter les données pour calculer des indices comme l'index de végétation différentiel normalisé (NDVI)\cite{NDVI}:
\begin{equation}
NDVI=\frac{NIR-VIS}{NIR+VIS}
\end{equation}
Où NIR correspond à l'intensité lumineuse dans le proche infrarouge et VIS dans le rouge. Une valeur de cet indice proche de 1 indique la présence de végétation, une valeur négative indique des nages ou l'absence de végétation. 
Dans notre cas, \textbf{les images multispectrales fournissent suffisamment d'informations par elles-mêmes pour classifier les sols avec précision}, sans passer par cet indice.

\paragraph{Algorithme SVM mieux adapté}
\paragraph{}
L'un des objectifs de notre projet étant de déterminer des principes généraux pour la classification et l'exploitation des données Sentinel-2, nous avons cherché à comparer les résultats obtenus pour différents classificateurs. Nous avons d'abord cherché à comparer à l'œil ces méthodes en cherchant des différences sur les images obtenues, comme sur la figure \ref{fig:comparaison}.

\begin{figure}[H]
  \centering
    \includegraphics[width=0.75\textwidth]{comparaison}
  \caption{Comparaison des classifications obtenues avec différentes méthodes.}
  \label{fig:comparaison}
\end{figure}
Cependant, ces comparaisons à l'œil ne permettent pas d'identifier formellement un critère meilleur que l'autre. Nous avons donc utilisé un critère numérique : l'exactitude, qui permet de quantifier la qualité de nos classifications et ainsi de les comparer. Les résultats sont résumés dans le tableau \ref{table:comp}. On peut constater que selon ces résultats, \textbf{le support vecteur machine (SVM) offre les meilleurs résultats} et semble donc le classificateur le plus adapté à notre problème de classification à trois classes. Il serait nécessaire de faire des tests similaires sur d'autres zones pour confirmer ou infirmer cette observation, cependant cela sort du cadre de notre projet.

\begin{table}[H]
\begin{center}
 \begin{tabular}{|c|c|c|c|c|}
  \hline
  type de classification & LDA & LSE & 1NN & SVM \\
  \hline
exactitude & 78.4\% & 90.9\% & 	98.0\% & 99.1\% \\
  \hline
  \end{tabular}
\end{center}
\caption{Comparaison des différents classificateurs}
\label{table:comp}
\end{table}

\paragraph{Importance de toutes les bandes spectrales}
\paragraph{} 
Nous avons également démontré que \textbf{chacune des 12 bandes spectrales est nécessaire dans la classification}, puisque la perte de l'une d'entre elles entraîne une baisse significative de l'exactitude comme on l'a vu dans le paragraphe \ref{lineaire}. Nous avons étudié l'influence de ce paramètre en étudiant l'exactitude d'un classificateur avec les 12 bandes, puis en en retirant. Les résultats obtenus sont présentés en figure \ref{fig:nBandes}. 
\newline
Il est à noter que n'avons utilisé que 12 bandes spectrales (sur les 13 fournies par le satellite) car l'une des bandes, la bande 11 à 1610nm, n'était  pas exploitable. Cela peut être dû à un défaut momentanément de fonctionnement du satellite. On peut donc espérer améliorer encore la qualité de nos résultats en utilisant les 13 bandes, une fois le problème réglé.

\begin{figure}[H]
  \centering
    \includegraphics[width=0.5\textwidth]{nb_bands}
  \caption{Influence du nombre de bandes spectrales sur la qualité de classification d'une méthode des 1 plus proches voisins.}
  \label{fig:nBandes}
\end{figure}


\subsection{Limites et améliorations}

\paragraph{Ajout d'une nouvelle classe}
\paragraph{}
Nous avons constaté que la boue le long des côtes n'est pas toujours reconnue de la même manière par nos algorithmes: parfois champs, parfois ville, ils ne correspondent à aucune de nos trois classes. Il est en effet rare en apprentissage automatique d'utiliser seulement trois classes. Nous avons donc rajouté une classe ``boue'' et testé certaines de nos classifications, en particulier le Support Vecteur Machine. Le résultat est donné en figure \ref{fig:SVM4Cl}. La boue est bien classifiée mais l'introduction de cette 4e classe entraîne l'apparition de faux positifs, en particulier dans les champs dont la réponse spectrale est proche.
\begin{figure}[H]
  \centering
    \includegraphics[width=0.75\textwidth]{SVM4Classes}
  \caption{Résultat de classification de 4 classes avec le support vecteur machine.}
  \label{fig:SVM4Cl}
\end{figure}

\paragraph{Pré-traitement des données}
\paragraph{}
Plusieurs méthodes s'offrent à nous pour améliorer ce résultat de classification: la plus simple, appelée \textit{whitening} (blanchissement, en français) consiste à centrer chaque bande spectrale sur sa moyenne et à la dilater proportionnellement à son écart-type. Cette méthode renormalise les distributions et permet une meilleure classification.

\paragraph{Ajout de nouvelles informations}
\paragraph{}
Il est également possible d'ajouter encore plus d'informations, en utilisant les données de satellites Sentinel-1 qui sont des images radar. Une dernière méthode tout juste mise au point par notre tuteur N. Brodu consiste à améliorer la résolution des bandes spectrales de plus faible résolution en inférant leur valeur à partir des bandes à plus haute résolution. Ces méthodes ont prouvé leur efficacité sur les données MODIS et pourraient apporter un supplément d'information sur les données Sentinel-2 également.

\paragraph{Dimensions spatiale et temporelle de notre travail}
\paragraph{}
Enfin, les données Sentinel-2 étant toutes nouvelles, peu de zones géographiques sont à notre disposition. Nous nous sommes concentrés pendant ce projet sur une zone précise dans un but de répétabilité. Nous avons essayé de classifier des zones différentes comme présenté en figure \ref{fig:zone2}, cependant ce travail n'a été qu'esquissé et sort du cadre de notre projet.

\begin{figure}[H]
  \centering
    \includegraphics[width=0.7\textwidth]{zone2}
  \caption{Résultat de classification d'une autre zone en Italie du nord}
  \label{fig:zone2}
\end{figure}

Il serait également intéressant d'effectuer un suivi temporel, mais là encore la mission Sentinel-2 est trop récente et les données disponibles ne sont pas suffisantes. 


\section{Bilan}

Ce projet nous a apporté de nouvelles expériences. D'un point de vue technique, nous avons tout particulièrement acquis des connaissances en machine learning. Sur un plan professionnel, nous avons été confrontés à des problématiques propres aux travaux en groupe : notamment à l’organisation et à la communication au sein du groupe. Nous avons ainsi acquis une méthode de travail et de gestion de projet (réunion hebdomadaire, dépôt des documents sur un répertoire commun, parallélisation des tâches ...). Par ailleurs, ce projet nous a permis de réfléchir aux notions de propriété intellectuelle et de valorisation d’un projet. Enfin, grâce à notre encadrant, Nicolas Brodu, nous avons pu approcher le monde de la recherche, d’une part parce que les réunions hebdomadaires avaient lieu à l’Inria, d’autre part au travers d'échanges plus informels avec le chercheur.

\paragraph{}
Nous tenons ainsi à remercier Nicolas Brodu pour son enthousiasme, son envie de transmettre et son soutien tout au long du projet.

\newpage
\bibliography{Biblio}
\bibliographystyle{plain}
\end{document}

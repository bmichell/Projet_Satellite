\label{ClassLin}
\begin{center}
 \begin{tikzpicture}
 \begin{scope}
 \draw[latex-latex, thin, draw=gray] (-4,0)--(4,0) node [right] {$x$}; % l'axe des abscisses
 \draw[latex-latex, thin, draw=gray] (0,-4)--(0,4) node [above] {$y$}; % l'axe des ordonnées
 \draw[thick] (-4,-4)--(4,4); % la courbe

\foreach \Point in {(-2,2), (-2,1.8), (-2,2.2), (-1.5,2), (-1.8,2.2),(-1.5,2.2),(-1.8,2),(-2.2,2),(-2,1.5),(-1.8,1.8)}{
    \node [blue] at \Point {\textbullet};
}

\foreach \Point in {(2,-2), (2,-1.8), (2,-2.2), (1.5,-2), (1.8,-2.2),(1.5,-2.2),(1.8,-2),(2.2,-2),(2,-1.5),(1.8,-1.8)}{
    \node [red] at \Point {\textbullet};
}

% to ensure that the points are being properly centered:
\draw [dotted, gray] (-4,-4) grid (4,4);
 \end{scope}
\end{tikzpicture}
\captionof{figure}{Illustration du principe de la classification linéaire.}
\end{center}

%opening

\section{Contexte}

Ce projet s'inscrit dans le cadre de notre formation ingénieur. Il s'étend sur six semaines et a été encadré par M. Brodu, chercheur post-doctorant à l'Inria dans l'équipe GeoStat (Géométrie et Statistiques dans les données d'acquisition) à Bordeaux. 

\subsection{Enjeux de la surveillance satellite}

Ce projet se situe dans un domaine particulier de l'utilisation des satellites orbitant autour de la Terre : la classification et le suivi des sols. 
Les applications concernent aussi bien l'agriculture que la surveillance environnementale (fonte des glaces, observation des océans et des forêts) en passant par la géologie, sans oublier la cartographie.
Les données satellties peuvent, par exemple, servir à analyser la qualité de l'air au niveau du sol et, entre autres, en déduire la présence d'évènements exceptionnels comme des feux de forêt ou du brouillard\cite{airSurv}. 

De manière plus concrète, M. Brodu, notre encadrant pédagogique, a mené des projets de traitement d'image satellite pour la surveillance environnementale. Il a notamment travaillé avec l'équipe OptIC (Optimal Inference in Complex and Turbulent Data), associée à l'Inria. L'idée était de suivre l'évolution de la végétation afin de repérer les zones de sécheresse, et ainsi pouvoir mieux gérer les ressources en eau. Une classification des sols de la région autour de Roorkee (Inde) a ainsi été effectuée. 
Cette classification a été réalisée grâce à des données d'imagerie multispectrale, qui est une technique de télédétection. C'est donc à cette technique que nous nous sommes intéressés, même s'il en existe d'autres (imagerie radar par exemple). 

\subsection{Imagerie satellite multispectrale}

Une image photographique couleur "classique" contient en réalité trois images : l'une dans le rouge, l'autre dans le vert et la troisième dans le bleu (RGB en anglais ou RVB en français). Elle ne contient donc que des informations dans le visible. 
Une image multispectrale, quant à elle, est formée de nombreuses images prises à des longueurs d'ondes variées. Elle contient donc plus d'informations, et selon les longueurs d'ondes d'acquisition peut avoir, en plus du visible, des informations dans l'ultraviolet ou l'infrarouge.

A partir de ces données multispectrales, il est possible de calculer des indices comme le NDVI (normalized difference vegetation index), permettant d'établir la présence ou l'absence de végétation dans une zone. En effet, les plantes chlorophyliennes absorbent fortement la lumière rouge et réfléchissent la lumière proche infrarouge : la différence normalisée d'intensité entre ces deux bandes correspond à l'indice NDVI. Une valeur proche de 1 indique la présence de végétation, et une valeur proche de 0 son absence. L'utilisation de cet indice est très courante, pour l'étude de la désertification de zones dues à l'activité humaine\cite{desert} par exemple.

Parmi les satellites utilisant cette technologie on trouve les satellites Aqua et Terra MODIS (Moderate-Resolution Imaging SpectroRadiometer). Ces deux satellites, lancés par la NASA en 1999 et 2002, imagent l'intégralité de la surface de la Terre tous les deux jours. Les données sont ensuite traitées et mises à disposition du public gratuitement, ce qui en a fait une des principales sources d'images satellites multispectrales. Ces satellites sont capables de mesurer 36 bandes fréquentielles avec trois résolutions différentes : 250m, 500m et 1000m par pixel\cite{nasa}. L'objectif de cette mission est de "jouer un rôle vital dans le développement de modèles globaux, validés et interactifs de systèmes terrestres capables de prévoir des changements globaux avec suffisamment de précision pour aider les décideurs politiques à prendre des décisions avisées concernant la protection de notre environnement."

La faible résolution des données MODIS ne permet en revanche pas de discerner des détails comme des cours d'eau ou des habitations, et des données plus précises sont donc nécessaires pour obtenir de meilleures classifications. 

Dans le cadre du projet \textit{Copernicus} d'observation de la Terre, l'agence spatiale européenne (ESA) développe en ce moment les missions Sentinel. Chacune de ces missions consiste en une paire de satellites en orbite autour de notre planète, récoltant des données sur la surface et l'atmosphère. Ainsi, les satellites Sentinel 2 (Sentinel 2-A lancé le 23 juin 2015 et 2-B dont le lancement est prévu pour la seconde moitié de 2016)\cite{sent2} récoltent des données multispectrales sur 13 bandes de fréquence : 4 bandes avec une résolution de 10m/pixel, 6 bandes à 20m/pixel et 3 à 60m/pixel.

Les résolutions de ces images satellites sont donc grandement meilleures que celles acquises par satellites MODIS. Leur mise à disposition du public étant elle aussi gratuite, on comprend leur intérêt. 
Néanmoins, la mission de l'Esa est récente : peu de données sont actuellement disponibles. La courverture temporelle, c'est-à-dire la fréquence à laquelle le satellite passe au-dessus d'une même zone, est également assez faible. Un seul des deux satellites ayant été lancé, le renouvellement des données se fait de manière hebdomadaire, alors que les satellites MODIS renouvellent les données deux fois par jour. 

\subsection{Classification et Machine Learning}

La classification de ces données multispectrales se fait à travers des algorithmes de "Machine Learning" (apprentissage automatique). 

\subsection{Motivation du projet et cahier des charges}

Les données Sentinel 2 étant très récentes, nous faisons partie des premiers à les analyser et aucun article à leur sujet n'a encore été publié. Leur analyse et leur classification relève donc de la recherche. 
Bien que ce projet s'étende sur une durée assez courte (6 semaines), les objectifs fixés étaient :
  - Définir les principaux intérêts des images multispectrales Sentinel 2
  - Etablir une classification de la surface terrestre (suivi d'occupation des sols) en trois classes ("Eau", "Urbain","Champ") à partir d'une image de la base de données Sentinel 2
  - Tester les algorithmes de Machine Learning les plus répandus
 
